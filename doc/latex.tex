% Use only LaTeX2e, calling the article.cls class and 12-point type.

\documentclass[12pt]{article}

% Users of the {thebibliography} environment or BibTeX should use the
% scicite.sty package, downloadable from *Science* at
% www.sciencemag.org/about/authors/prep/TeX_help/ .
% This package should properly format in-text
% reference calls and reference-list numbers.

%\usepackage{scicite}
\usepackage{natbib}

\usepackage{amssymb} %% added for \dashleftarrow and \dashrightarrow
\usepackage{graphicx} %% added for figures
%\usepackage{color} %% added for colors
\usepackage[dvipsnames]{xcolor} %% added for colors
%\usepackage[dvips]{color} %\textcolor[rgb]{1,0,0}{...}
\usepackage{amsmath, amsthm, amssymb} 
\usepackage[ruled]{algorithm2e}
\usepackage{bm} 
\usepackage{mathrsfs}

% Use times if you have the font installed; otherwise, comment out the
% following line.

\usepackage{times}

% for package description
\usepackage{dirtree}

%\usepackage{pdfpages}
\usepackage{hyperref}
%\hypersetup{urlcolor=blue, colorlinks=true} % Colors hyperlinks in blue - change to black if annoying
\hypersetup{
    colorlinks=true,
    linkcolor=NavyBlue,
    %linkcolor=Blue,
    %linkcolor=blue,
    %linkcolor=MidnightBlue,
    citecolor=violet,
    filecolor=magenta,      
    urlcolor=cyan,
}

% --- UPDATE SA --- 18072016
% To make link on the page number only in table of content
\hypersetup{linktocpage}
% Provides the \newgeometry to change the margin of one page
% I use it to fit the S1 figure in a page with the section name without reducing the S1 size
\usepackage{geometry}
% (1) --> \newgeometry{left=3cm,bottom=0.1cm}
% your page...
% (2) --> \restoregeometry
% --- UPDATE SA --- 18072016

%\startlocaldefs
\usepackage{centernot}% http://ctan.org/pkg/centernot
\newcommand{\adjacent}[1][.7em]{\mathrel{\rule[.5ex]{#1}{.4pt}}}
\newcommand{\notadjacent}[1][.7em]{\mathrel{\centernot{\adjacent[#1]}}}

\makeatletter
%\renewcommand{\fnum@figure}{\figurename~S\thefigure}
%\renewcommand{\fnum@table}{\tablename~S\thetable}
\renewcommand{\thefigure}{S\arabic{figure}} 
\renewcommand{\thetable}{S\arabic{table}} 
\renewcommand{\theequation}{S\arabic{equation}} 
\makeatother


\topmargin 0.0cm
\oddsidemargin 0.2cm
\textwidth 16cm 
\textheight 21cm
\footskip 1.0cm


\newenvironment{sciabstract}{%
\begin{quote} \bf}
{\end{quote}}


\newcounter{lastnote}
\newenvironment{scilastnote}{%
\setcounter{lastnote}{\value{enumiv}}%
\addtocounter{lastnote}{+1}%
\begin{list}%
{\arabic{lastnote}.}
{\setlength{\leftmargin}{.22in}}
{\setlength{\labelsep}{.5em}}}
{\end{list}}


\renewcommand\refname{} %remove References to be replaced by Supplementary References section .

% Include your paper's title here

%\title{A simple {\it Science\/} Template} 
%\title{Causal network reconstruction when all variables are not observed} 
%\title{Causal network reconstruction from large genomic datasets with unobserved latent variables} 
%\title{Causal network reconstruction with latent variables, applications to large genomic datasets} 
%\title{Learning causal networks with latent variables\\ from large-scale genomic data} 
%\title{Information-theoretic learning of causal networks with latent variables for large genomic data} %94
%\title{Information-theoretic learning of causal networks from large genomic data with latent variables} %95
%\title{Information-theoretic learning of causal networks with latent variables for genomic data} %88
%\title{Information theoretic reconstruction of causal networks with latent variables for genomic data} %94
\title{Supplementary Information\\
{\small \em on the manuscript}\\
Learning causal networks with latent variables\\ from multivariate information in genomic data} %92



% Place the author information here.  Please hand-code the contact
% information and notecalls; do *not* use \footnote commands.  Let the
% author contact information appear immediately below the author names
% as shown.  We would also prefer that you don't change the type-size
% settings shown here.

\author
{Nadir Sella,$^{1,2}$ Louis Verny,$^{1,2,3}$ S\'everine Affeldt,$^{1,2,4}$\\  Herv\'e Isambert,$^{1,2,\dag}$\\
%\normalsize{$^{1}$Institut Curie, PSL University, UPMC,}\\
%\normalsize{26 rue d'Ulm, 75005 Paris, France}\\
\\
\normalsize{$^{1}$Institut Curie, PSL Research University, CNRS, UMR168, 26 rue d’Ulm, 75005 Paris, France}\\
\normalsize{$^{2}$Sorbonne Universit\'es, UPMC Univ Paris 06, 4, Place Jussieu, 75005 Paris, France}\\
\\
\normalsize{$^{3}$ Current address: CENOVA SAS, 100 Avenue Charles De Gaulle, 92200 Neuilly Sur Seine}\\
\normalsize{$^{4}$ Current address: ICAN, UPMC, Paris, France}\\
\\
\normalsize{$^\dag$To whom correspondence should be addressed; E-mail:  herve.isambert@curie.fr}
}

% Include the date command, but leave its argument blank.

\date{}








\title{User guide for the MIIC algorithm\\command line version}






\begin{document}
\maketitle

The main folder contains the scripts and source code for the reconstruction of networks starting from observation data. The structure is the following:


The directories are organized as follow:
\dirtree{%
.1 /.
.2 common.
.3 \textit{miic.R}.
.3 \textit{gmPlot.R}.
.3 \textit{gmSummary.R}.
.2 sharedLib.
.2 data.
.3 \textit{some data input/output...}.
.2 miic.
.3 \textit{all miic executables}.
.2 doc.
.3 \textit{documentation}.
}
%\pagebreak
%----------------------------------------------------------------------------------------
%	SECTION 2
%----------------------------------------------------------------------------------------
\subsection*{Package requirements}
To launch the miic.R script you need to have R installed on your machine, along with some packages that are available in the CRAN repository. 
\begin{description}
		\item[Rpackages]	MASS, getopt, plotrix, methods, igraph, ppcor, bnlearn   
\end{description}
	
\subsection*{Calling the inference methods with {\tt miic}}

You can call the inference methods through the \textit{miic.R} script.

\begin{itemize}

	\item[] \underline{Overview}

	\begin{description}
		\item[main] \textit{$\sim$/common/miic.R}
		\item[lib]	\textit{$\sim$/common/lib/...}
		%\item[Rpackages]	getopt, pcalg, minet, bnlearn, infotheo
	\end{description}

	\item[] \underline{Arguments} \textit{\small{(mandatory: *)}}
	
	\begin{description}
		\item[-i]* file path of the input dataset\footnote{The input dataset should be a {\color{black}{tab separated table}}, with column names but no row names. Missing values should be indicated with \textit{NA}. Each column corresponds to a categorized variable and each row to one sample.}
		\item[-o]* directory path for the output of the inference method\footnote{To prevent from overwriting existing results, if the output directory already exists, the skeleton inference step returns a message and stops.}
		\item[-d] {\color{black}{steps to perform (`1,2,3,4' or `1,2' or `1,3' \textit{etc...})\\ default: `1,2,3,4'}}\footnote{{\color{black}{(1) skeleton, (2) orientation, (3) summary, (4) plot}}}
		\item[-p] parameters for the inference method (see the following subsections). The value expected here is of type character: `$param_{1}$:$value_{1}$,$param_{2}$:$value_{2}$ \textit{etc...}'
		\item[-t] file path to the true edges; used during the \textit{summary} step\footnote{The true edges file has two space-separated columns. Each line corresponds to one edge. The orientation is $col1 \rightarrow col2$.}
		\item[-s] file path to the category order used during the \textit{summary} step\footnote{The orientation is $col1 \rightarrow col2$. This files provides information about how to consider the different states of categorical variables. It will be used to compute the signs of the edges (using spearman correlation coefficient) by ranking the levels of each variable according to the order given in the file. This file is necessary (except for numerical variables) to obtain edge colors corresponding to the signs of their partial correlations (positive in red, negative in blue). 
        %in the plots shown in the web server. 
        If it is not possible or desirable to order the states of some variables, the column ``levels\_increasing\_order'' can be left empty for these variables. The edges involving those variables are then colored in gray in the reconstructed network. 
        %giving the possibility to obtain a partially colored network. 
        (NB: in this case, the field separator is still needed between the node name and the empty ``levels\_increasing\_order'' cell in the category order file).}
        \item[-x] file path to the blackbox file containing the edges to be removed at the beginning of the reconstruction step.\footnote{It must be formatted as a two-column file, {\tt Node1 Node2}, with a field separator between them.}
		\item[-l] file path to the layout of each vertex; used during the \textit{plot} step\footnote{The layout file has three {\color{black}{tab}} separated columns, the first column being optional. Each line corresponds to the (x,y) coordinates of each vertex. The first column can contain the label of the vertex as indicated in the colnames of the input dataset table. The order in which the coordinates are given also corresponds to the order of the colnames of the input dataset table.}
		\item[-c] if given, edges will be filtered according to their {\color{black}{confidence ratio}}. It needs two parameters, described in \ref				{sssec:optionC}. To use a different threshold without shuffling the data a second time, an identical command line with a different {\color{black}{confidence ratio}} threshold can be used. The program will keep the edges satisfying the new {\color{black}{confidence ratio}}, using the previously calculated mutual information
		\item[-v] if given, detailed information is given in the log file and command shell
		\item[-n] if given, the graphml file is created in the output folder\footnote{If the layout file is provided, the network will be stored in a xgmml file format, that allows the node positioning in the Cytoscape tool. We have noticed that in this case edges are note coloured, we will soon try to solve it.}
	\end{description}

\end{itemize}

\underline{An example}:\\
\textit{\small{
Rscript \textbf{miic.R -i } ../data/asiaDataset.txt \textbf{-o} ../data/asiaNetwork \textbf{-m} miic \textbf{-p} cpx:nml,efn:-1,lat:yes,prg:yes  \textbf{-c} csh:100,ccr:0.01
}}
\\

When calling the available inference methods with \emph{miic.R}, the `p' option can be used to indicate the chosen parameters. The value expected for this option is of type character:\\ `$param_{1}$:$value_{1}$,$param_{2}$:$value_{2}$ \textit{etc...}'. The possible $param_{i}$ and $value_{i}$ for each method are detailed in the following subsections.

\subsubsection*{Option `-p' for {\tt\bfseries miic}}

\begin{description}
	\item[cpx] formula used to compute the complexity term [`mdl'\footnote{Maximum description length} or `nml'\footnote{Normalized Maximum Likelihood}]\\default:~nml (\textit{\small{Ex.:~-p `...,cpx:mdl,...'}})
%	\item[eta] likelihood ratio is compared to $\eta$ for the edge removal decision\\default:~1 (\textit{\small{Ex.:~-p ...,eta:5,...}})
	\item[lat] should the network be reconstructed under the hypothesis that some variables might not be observed? 
	[`yes' or `no'] \\default:~no (\textit{\small{Ex.:~-p ...,lat:yes,...}})	
	\item[prg] should the network be oriented using the propagation rule? 
	[`yes' or `no'] \\default:~yes (\textit{\small{Ex.:~-p ...,prg:yes,...}})
	
	
	\item[efn] number of uncorrelated samples\\default:~number of rows of the input dataset (\textit{\small{Ex.:~-p ...,efn:1000,...}})

\end{description}

\underline{A `-p' example}:~\textit{\small{-p cpx:mdl,efn:1000}}


\subsubsection*{Option `-c' for {\tt\bfseries miic}}\label{sssec:optionC}
\begin{description}
	\item[csh] number of random shuffling of the input dataset, in order to get the random mutual information between {\tt miic} inferred edges  		(\textit{\small{Ex.:~-c ...,csh:10,...}})
	\item[ccr] {\color{black}{confidence ratio}} used as a threshold for filtering the edges. (\textit{\small{Ex.:~-c ...,ccr:0.01,...}})
\end{description}
\underline{A `-c' example}:~\textit{\small{-c csh:100,ccr:0.01}}

\label{subsec:directCall}



\subsubsection*{Viewing inferred networks}
\label{subsec:results}

The inferred networks can be viewed in pdf format,automatically generated with igraph ({\href{http:
//igraph.org/}{http://igraph.org/}) or manipulated for better display using cytoscape (\href{http://
www.cytoscape.org}{http://www.cytoscape.org}). The files are located in the following directories:
\begin{itemize}
\item[$\bullet$] Unfiltered network: the pdf and graphml files are located in the output directory set by the -o entry in the command line
\item[$\bullet$] Filtered networks (using -c option with {\tt miic}): the pdf and graphml file are located in the subdirectory `shuffle\_[cshValue]/filtered\_network\_[ccrValue]', which can be found in the output directory set by the -o entry in the command line
\end{itemize}

In order to visualize the network in a correct, pleasant and interactive way, we recommend the utilization of Cytoscape tool, version 3.1.0 or later (\url{http://www.cytoscape.org/}). Cytoscape is available for Windows, Linux and OsX. \\In order to load the network you have to go through the following steps:

\begin{enumerate}
\item Import the network: File$\Rightarrow$Import$\Rightarrow$Network$\Rightarrow$File, and select the graphml file
\item Import the style: File$\Rightarrow$Import$\Rightarrow$Styles, and select the miic\_style.xml file present in this folder
\item Select the loaded style: under the ``Style'' panel present in ``Control Panel'' select miic\_style 

\end{enumerate}

\end{document}


